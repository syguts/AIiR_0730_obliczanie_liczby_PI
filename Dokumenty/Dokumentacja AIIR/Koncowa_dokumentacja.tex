%%%%%%%%%%%%%%%%%%%%%%%%%%%%%%%%%%%%%%%%%%%%%%%%%%%%%%%%%%%%%%%%%%%%%%%%%%%%%%%%%%%%%%%%%
% Projekt: AIiR																		%
% Semestr VI																			%
% Data: Marzec / Czerwiec 2015															%
%%%%%%%%%%%%%%%%%%%%%%%%%%%%%%%%%%%%%%%%%%%%%%%%%%%%%%%%%%%%%%%%%%%%%%%%%%%%%%%%%%%%%%%%%

%----------------------------------------------------------------------------------------
% PREAMBUŁA: PAKIETY I KONFIGURACJA DOKUMENTU
%----------------------------------------------------------------------------------------

\documentclass[a4paper,12pt]{article}		% klasa dokumentu i rozmiar czcionki
\usepackage[utf8]{inputenc}					% kodowanie uniwersalne
\usepackage[margin=2.5cm]{geometry}			% określa wielkość marginesów
\usepackage{polski}							% polski pakiet językowy
\usepackage{fancyhdr}						% nagłówek i stopka
\usepackage{lastpage}						% potrzebny fancyhdr do numerowania
\usepackage{extramarks}						% potrzebny fancyhdr
\usepackage{amsmath}						% wzory matematyczne
\usepackage{relsize}						% zmiana wielkości wzorów matematycznych
\usepackage{booktabs}						% do tabel
\usepackage{multirow}						% do łączenia wierszy w tabeli
\usepackage{multicol}						% do łączenia kolumn w tabeli
\usepackage{placeins}						% do stawiania "barier" tabeli
\usepackage{graphicx}						% załączanie obrazów
\usepackage{caption}						% zarządzanie podpisami pod floatami
\usepackage{gensymb}						% znaki specjalne
\usepackage[parfill]{parskip}				% dodaje pustą linię między paragrafami
\usepackage{hyperref}						% links
\usepackage{url}							% generowanie linków ze znakami specjalnymi
\usepackage{pdflscape}						% zmiana orientacji strony
\usepackage{indentfirst}					% dodanie pierwszego wcięcia w tekście
\usepackage{listings}
%----------------------------------------------------------------------------------------
% PREAMBUŁA: DEFINICJE KOMED DLA NAGŁÓWKA I STOPKI
%----------------------------------------------------------------------------------------

\newcommand{\AuthorName}{Sternik,Cichuta,Cebula,Sygut}
\newcommand{\Title}{Obliczanie przybliżenia liczby Pi.}
\newcommand{\UnderTitle}{Dokumentacja końcowa}
\newcommand{\University}{Politechnika Wrocławska}
\newcommand{\Class}{Aplikacje Internetowe i Rozproszone}
\newcommand{\ClassDay}{Środa}
\newcommand{\ClassTime}{07:30}
\newcommand{\ClassInstructor}{dr inż.~Marek Woda}

%----------------------------------------------------------------------------------------
% PREAMBUŁA: DEFINICJE KOMEND
%----------------------------------------------------------------------------------------

\renewcommand{\baselinestretch}{1.20}				% wielkość odstępu między wierszami
\renewcommand\headrulewidth{0.4pt}					% wielkość nagłówka
\renewcommand\footrulewidth{0.4pt}					% wielkość stopki
\newcommand{\HRule}{\rule{\linewidth}{0.5mm}}		% szerokość poziomego paska

%----------------------------------------------------------------------------------------
% PREAMBUŁA: USTAWIENIA NAGŁÓWKA I STOPKI
%----------------------------------------------------------------------------------------

\pagestyle{fancy}										% styl strony stosujący fancyhdr
\lhead{Obliczanie liczby Pi}							% górny-lewy
\rhead{\Class}											% górny-prawy
\rfoot{Strona\ \thepage\ z\ \protect\pageref{LastPage}}	% dolny-prawy

%----------------------------------------------------------------------------------------
% PREAMBUŁA: INNE USTAWIENIA
%----------------------------------------------------------------------------------------

\setlength{\parindent}{1cm} 							% szerokość wcięć w paragrafach

%----------------------------------------------------------------------------------------
% PDF META INFO
%----------------------------------------------------------------------------------------

\pdfinfo
{
	/Author (\AuthorName)
	/Title (\Class \Title)
	/CreationDate (\today)
	/Subject (\Class)
	/Keywords (\Class, \Title, LaTeX)
}

%----------------------------------------------------------------------------------------
% STRONA TYTUŁOWA: SEKCJA NAGŁÓWKA
%----------------------------------------------------------------------------------------

\begin{document}
\begin{titlepage}
\hfill Wrocław, dn. 10 czerwca 2015r.\\
\center
\textsc{}\\[1.5cm]
\textsc{\LARGE \University}\\[1.5cm]
\textsc{\Large \Class}\\[1.5cm]

%----------------------------------------------------------------------------------------
% STRONA TYTUŁOWA: SEKCJA TYTUŁU
%----------------------------------------------------------------------------------------

\HRule \\[0.7cm]
{ \huge \bfseries \Title}\\[0.4cm]
\textsc{\large \UnderTitle}\\[0.5cm]
\HRule \\[1.0cm]

%----------------------------------------------------------------------------------------
% STRONA TYTUŁOWA: SEKCJA AUTORA
%----------------------------------------------------------------------------------------

\begin{minipage}{0.5\textwidth}
\begin{flushleft} \large
\emph{Grupa projektowa:}
\\ Paweł \textsc{Sternik}, 200623
\\ Kamil \textsc{Cichuta}, ???
\\ Mariusz \textsc{Cebula}, ???
\\ Sławomir \textsc{Sygut}, ???
\end{flushleft}
\end{minipage}
~
\begin{minipage}{0.4\textwidth}
\begin{flushright} \large
\emph{Prowadzący:}
\\ dr inż.~Marek \textsc{Woda}
\end{flushright}
\end{minipage}\\[3cm]

%----------------------------------------------------------------------------------------
% STRONA TYTUŁOWA: SEKCJA DATY
%----------------------------------------------------------------------------------------

\emph{Termin spotkań:}\\[0.35cm]
{\large \ClassDay, godz. \ClassTime}\\[0.5cm]
\vfill
\end{titlepage}

%----------------------------------------------------------------------------------------
% SPIS TREŚCI
%----------------------------------------------------------------------------------------

\newpage
\tableofcontents
\newpage

%----------------------------------------------------------------------------------------
% DOKUMENT: SEKCJA
%----------------------------------------------------------------------------------------

\section{Temat projektu.}
\section{Cel projektu.}
\subsection{Cel dydaktyczny.}
\subsection{Cel merytoryczny.}
\section{Opis zastosowanego algorytmu.}
\section{Zastosowane technologie.}
\subsection{Framework Django.}
\subsection{Technologia MPI.}
\subsection{CSS i HTML.}
\subsection{Komunikacja grupy.}
\subsubsection{GitHub.}
\subsubsection{Trello.}
\section{Plan realizacji.}
\subsection{Podział pracy między członków grupy.}
\subsection{Terminarz realizacji zadań.}
\section{Implementacja silnika obliczeniowego.}
\section{Aplikacja internetowa.}
\section{Testy.}
\section{Podsumowanie i wnioski.}

\end{document}
