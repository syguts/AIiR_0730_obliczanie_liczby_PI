%%%%%%%%%%%%%%%%%%%%%%%%%%%%%%%%%%%%%%%%%%%%%%%%%%%%%%%%%%%%%%%%%%%%%%%%%%%%%%%%%%%%%%%%%
% Projekt: AIiR																		%
% Semestr VI																			%
% Data: Marzec / Czerwiec 2015												%
%%%%%%%%%%%%%%%%%%%%%%%%%%%%%%%%%%%%%%%%%%%%%%%%%%%%%%%%%%%%%%%%%%%%%%%%%%%%%%%%%%%%%%%%%

%----------------------------------------------------------------------------------------
% PREAMBUŁA: PAKIETY I KONFIGURACJA DOKUMENTU
%----------------------------------------------------------------------------------------

\documentclass[a4paper,12pt]{article}		% klasa dokumentu i rozmiar czcionki
\usepackage[utf8]{inputenc}					% kodowanie uniwersalne
\usepackage[margin=2.5cm]{geometry}			% określa wielkość marginesów
\usepackage{polski}							% polski pakiet językowy
\usepackage{fancyhdr}						% nagłówek i stopka
\usepackage{lastpage}						% potrzebny fancyhdr do numerowania
\usepackage{extramarks}						% potrzebny fancyhdr
\usepackage{amsmath}						% wzory matematyczne
\usepackage{relsize}						% zmiana wielkości wzorów matematycznych
\usepackage{booktabs}						% do tabel
\usepackage{multirow}						% do łączenia wierszy w tabeli
\usepackage{multicol}						% do łączenia kolumn w tabeli
\usepackage{placeins}						% do stawiania "barier" tabeli
\usepackage{graphicx}						% załączanie obrazów
\usepackage{caption}						% zarządzanie podpisami pod floatami
\usepackage{gensymb}						% znaki specjalne
\usepackage[parfill]{parskip}				% dodaje pustą linię między paragrafami
\usepackage{hyperref}						% links
\usepackage{url}							% generowanie linków ze znakami specjalnymi
\usepackage{pdflscape}						% zmiana orientacji strony
\usepackage{indentfirst}					% dodanie pierwszego wcięcia w tekście
\usepackage{listings}
\usepackage{color}
\usepackage{array}
 
\definecolor{codegreen}{rgb}{0,0.6,0}
\definecolor{codegray}{rgb}{0.5,0.5,0.5}
\definecolor{codepurple}{rgb}{0.58,0,0.82}
\definecolor{backcolour}{rgb}{0.95,0.95,0.92}
 
\lstdefinestyle{mystyle}{
    backgroundcolor=\color{backcolour},   
    commentstyle=\color{codegreen},
    keywordstyle=\color{magenta},
    numberstyle=\tiny\color{codegray},
    stringstyle=\color{codepurple},
    basicstyle=\footnotesize,
    breakatwhitespace=false,         
    breaklines=true,                 
    captionpos=b,                    
    keepspaces=true,                 
    numbers=left,                    
    numbersep=5pt,                  
    showspaces=false,                
    showstringspaces=false,
    showtabs=false,                  
    tabsize=2
}
 
 \lstset{style=mystyle}
%----------------------------------------------------------------------------------------
% PREAMBUŁA: DEFINICJE KOMED DLA NAGŁÓWKA I STOPKI
%----------------------------------------------------------------------------------------

\newcommand{\AuthorName}{Sternik,Cichuta,Cebula,Sygut}
\newcommand{\Title}{Obliczanie przybliżenia liczby Pi.}
\newcommand{\UnderTitle}{Dokumentacja końcowa}
\newcommand{\University}{Politechnika Wrocławska}
\newcommand{\Class}{Aplikacje Internetowe i Rozproszone INES00304P}
\newcommand{\ClassDay}{Środa}
\newcommand{\ClassTime}{07:30}
\newcommand{\ClassInstructor}{dr inż.~Marek Woda}

%----------------------------------------------------------------------------------------
% PREAMBUŁA: DEFINICJE KOMEND
%----------------------------------------------------------------------------------------

\renewcommand{\baselinestretch}{1.20}				% wielkość odstępu między wierszami
\renewcommand\headrulewidth{0.4pt}					% wielkość nagłówka
\renewcommand\footrulewidth{0.4pt}					% wielkość stopki
\newcommand{\HRule}{\rule{\linewidth}{0.5mm}}		% szerokość poziomego paska

%----------------------------------------------------------------------------------------
% PREAMBUŁA: USTAWIENIA NAGŁÓWKA I STOPKI
%----------------------------------------------------------------------------------------

\pagestyle{fancy}										% styl strony stosujący fancyhdr
\lhead{Obliczanie liczby Pi}							% górny-lewy
\rhead{\Class}											% górny-prawy
\rfoot{Strona\ \thepage\ z\ \protect\pageref{LastPage}}	% dolny-prawy

%----------------------------------------------------------------------------------------
% PREAMBUŁA: INNE USTAWIENIA
%----------------------------------------------------------------------------------------

\setlength{\parindent}{1cm} 							% szerokość wcięć w paragrafach

%----------------------------------------------------------------------------------------
% PDF META INFO
%----------------------------------------------------------------------------------------

\pdfinfo
{
	/Author (\AuthorName)
	/Title (\Class \Title)
	/CreationDate (\today)
	/Subject (\Class)
	/Keywords (\Class, \Title, LaTeX)
}

%----------------------------------------------------------------------------------------
% STRONA TYTUŁOWA: SEKCJA NAGŁÓWKA
%----------------------------------------------------------------------------------------

\begin{document}
\begin{titlepage}
\hfill Wrocław, dn. 10 czerwca 2015r.\\
\center
\textsc{}\\[1.5cm]
\textsc{\LARGE \University}\\[1.5cm]
\textsc{\Large \Class}\\[1.5cm]

%----------------------------------------------------------------------------------------
% STRONA TYTUŁOWA: SEKCJA TYTUŁU
%----------------------------------------------------------------------------------------

\HRule \\[0.7cm]
{ \huge \bfseries \Title}\\[0.4cm]
\textsc{\large \UnderTitle}\\[0.5cm]
\HRule \\[1.0cm]

%----------------------------------------------------------------------------------------
% STRONA TYTUŁOWA: SEKCJA AUTORA
%----------------------------------------------------------------------------------------

\begin{minipage}{0.5\textwidth}
\begin{flushleft} \large
\emph{Grupa projektowa:}
\\ Paweł \textsc{Sternik}, 200623
\\ Kamil \textsc{Cichuta}, 200838
\\ Mariusz \textsc{Cebula}, 20772
\\ Sławomir \textsc{Sygut}, 200668
\end{flushleft}
\end{minipage}
~
\begin{minipage}{0.4\textwidth}
\begin{flushright} \large
\emph{Prowadzący:}
\\ dr inż.~Marek \textsc{Woda}
\end{flushright}
\end{minipage}\\[3cm]

%----------------------------------------------------------------------------------------
% STRONA TYTUŁOWA: SEKCJA DATY
%----------------------------------------------------------------------------------------

\emph{Termin spotkań:}\\[0.35cm]
{\large \ClassDay, godz. \ClassTime}\\[0.5cm]
\vfill
\end{titlepage}

%----------------------------------------------------------------------------------------
% SPIS TREŚCI
%----------------------------------------------------------------------------------------

\newpage
\tableofcontents
\newpage

%----------------------------------------------------------------------------------------
% DOKUMENT: SEKCJA
%----------------------------------------------------------------------------------------

\section{Temat projektu.}
Tematem projektu realizowanego w ramach kursu Aplikacje Internetowe i Rozproszone było wyznaczanie rozszerzenia liczby Pi z wykorzystaniem algorytmu Monte Carlo. Realizacja wymagała implementacji kilku modułów stanowiących cały system.

\section{Cel projektu.}

\subsection{Cel dydaktyczny.}
Głównym celem dydaktycznym kursu było zapoznanie się z technologiami wykorzystywanymi do tworzenia rozproszonych aplikacji połączonych z internetowymi klientami. Ponadto kurs wymagał zoorganizowania pracy w grupach kilkuosobowych co wymuszało zaplanowanie kolejnych etapów pracy i podział poszczególnych zadań między członków grupy. Wykorzystane zostały do tego odpowiednie narzędzia komunikacji, które zostaną opisane dokładniej w dalszej częsci dokumentu. 

\subsection{Cel merytoryczny.}
Implementacja algorytmu Monte Carlo obliczającego rozszerzenie liczby Pi z wykorzystaniem technologi MPI (ang. Message Passing Interface ) czyli protokołu komunikacyjnego służącego do przesyłania komunikatów pomiędzy procesami programów równoległych. Ponadtwo stworzenie aplikacji internetowej która poprzez stworzoną bazę dancyh łączy się z silnikiem obliczeniowym działającym na kilku niezależnych komputerach. Strona powinna posiadać elementy zmieniające się dynamicznie podczas realizacji zadania - przykładowo pasek postępu. 
\newpage

\section{Opis zastosowanego algorytmu.}

\subsection{Opis słowny algorytmu.}
Metoda Monte Carlo stosowana jest do problemów, które jest bardzo trudno rozwiązać za pomocą podejścia analitycznego. Najczęściej stosuje się ją do modelowania złożonych problemów takich jak:
 
\begin{enumerate}
\item Obliczanie całek.
\item Obliczanie łańcuchów procesów statystycznych .
\item Obliczanie złożonych symulacji.
\end{enumerate}

Metoda opiera się na losowaniu nazywanym w tym przypadku wyborem przypadkowym. Losowanie dokonywane jest zgodnie z rozkładem, który jest znany. Przykładowo całkowanie metodą Monte-Carlo działa na zasadzie porównywania losowych próbek z wartością funkcji. Dokładność wyniku uzyskanego tą metodą jest zależna od liczby sprawdzeń i jakości użytego generatora liczb pseudolosowych. Zwiększanie liczby prób nie zawsze zwiększa dokładność wyniku, ponieważ generator liczb pseudolosowych ma skończenie wiele liczb losowych w cyklu. Przykładowo całkowanie tą metodą jest używane w~przypadkach, kiedy szybkość otrzymania wyniku jest ważniejsza od jego dokładności (np. obliczenia inżynierskie).\\

W projekcie algorytm Monte Carlo zostanie wykorzystany do obliczenia rozszerzenia dziesiętnego liczby Pi. Dokładny opis realizacji tego algorytmu znajduję się w następnym punkcie.

\subsection{Obliczanie liczby Pi metodą Monte Carlo.} 
Liczbę Pi można obliczać na wiele różnych sposobów. Jednym z nich jest wykorzystanie metody Monte Carlo. Metoda ta charakteryzuję się przede wszystkim swoją stosunkową prostą procedurą jak i przystosowaniem do zaimplementowania mechanizmów zrównoleglenia - co jest jednym z głównych celów projektu.

\begin{center}
\begin{figure}[h!]
\centering
\includegraphics[scale=0.5]{Resources/Kolo_rysunek.png}
\caption{Koło o promieniu r wpisane w kwadrat - bok 2r.}
\end{figure}
\end{center}

Pole kwadratu przedstawionego na Rysunku 1 wynosi:
 \[(4\prod)^{2}\]
 natomiast koła oczywiście:
 \[(\prod)r^{2}\]
 \newpage
 
 Znając zatem pole koła jak i kwadratu można zauważyć że stosunek pola koła do pola kwadratu wynosi:
\[ \frac{PoleKola}{PoleKwadratu}=\frac{\prod r^{2}}{4r^{2} }= \frac{\prod}{4}\]
Następnie korzystając z tego, że pole koła jak i kwadratu zostały w jakiś sposób obliczone można wyciągnąć wzór na liczbę Pi:
\[\prod =4\frac{PoleKola}{PoleKwadratu}\]
Dojście do tego punktu mogłoby się wydawać zatoczeniem koła i powrotem do punktu początkowego problemu. Tak nie jest. Ponieważ dopiero w tym miejscu pojawia się cała charakterystyka metody Monte Carlo. Abstrakcyjnie należy sobię teraz wyobraźić sytuację, w której rzucamy bardzo dużo razy rzutkami w tarczę wyglądająca jak Rysunek 1. Kończąc grę, plansza będzie pokryta rzutkami znajdującymi sie we wnętrzu koła jak i poza nim na obszarze kwadratu. Podsumowując liczba rzutek w i poza kołem będzie równa stosunkowi pola koła do pola kwadratu.

\begin{center}
\begin{figure}[h!]
\centering
\includegraphics[scale=0.7]{Resources/Losowanie.png}
\caption{Przykład losowania punktów.}
\end{figure}
\end{center}

 Formalnie losowanie będzie odbywać się pośród punktów należących do zbioru pokrytego współrzędnymi należacymi do predziału [-2r,2r]. Stosunek liczby punktów
zawierających się w kole o środku w punkcie [0,0] i promieniu r do wszystkich wylosowanych punktów będzie dążył w nieskończoności (z pewnym prawdopodobieństwem) do stosunku tego pola koła do koła kwadratu o boku 2r. Co więcej, stosunek ten będzie identyczny również do ćwiartki koła. Jeżeli pole koła podzielimy na cztery i tak samo podzielimy pole kwadratu, to ich stosunek będzie wciąż taki sam. Oznacza to, że wystarczy, jeżeli będziemy losowali punkty o współrzędnych od 0 do r. Cała metoda sprowadza się więc do tego, by losować punkty, sprawdzać, czy mieszczą się w kole, i następnie podstawiać liczby wylosowanych punktów do wzoru. Losując odpowiednio dużo punktów, powinniśmy otrzymać z pewnym prawdopodobieństwem rozsądne przybliżenie liczby Pi.

\subsection{Implementacja algorytmu - program jednowątkowy.} 
Dla lepszego zrozumienia działania algorytmu i sprawdzenia jego rezultatów stworzony został program w języku C++ obliczający przybliżenie liczby Pi wykonujący wszystkie obliczenia szeregowo, czyli po prostu korzystający z jednego wątku. Program przyjmuje od użytkownika zadaną liczbę punktów. W rezultacie wyświetla obliczoną liczbę Pi oraz orginalne rozwinięcie.

\lstinputlisting[language=C++]{Code/obliczanie_pi.cpp}

\begin{figure}[h!]
\centering
\includegraphics[scale=0.6]{Resources/Screen_Program_JedenWatek}
\caption{Przykład działania programu jednowątkowego.} 
\end{figure}

Jak widać na zamieszczonym zdjęciu ekranu program jednowątkowy obliczający rozwinięcie dziesiętne liczby Pi korzystając z metody Monte Carlo obliczył liczbę Pi równą 3.14215. Otrzymana dokładność sięga jedynie dwóch miejsc po przecinku przy stosunkowo już dużej ilości losowanych punktów - 10 mln. Program wyświetlił informację o tym, że 7855365 punktów znalazło się w kole.
\newpage
\section{Zastosowane technologie.}

\subsection{Framework Django.}

\begin{figure}[h!]
\centering
\includegraphics[scale =0.2]{Resources/django.png}
\caption{Logo frameworku Django.} 
\end{figure}

\subsubsection{Opis}
Django – wysokopoziomowy, opensource’owy framework przeznaczony do tworzenia aplikacji internetowych, napisany w Pythonie. Powstał pod koniec 2003 roku jako ewolucyjne rozwinięcie aplikacji internetowych, tworzonych przez grupę programistów związanych z Lawrence Journal-World. Nazwa frameworku pochodzi od imienia gitarzysty Django Reinhardta.
Framework wymaga instalacji interpretera Pythona. W ramach projektu wykorzystany został Python 2.7.6, instalacja jednak nie była konieczna, ponieważ w systemie Linux był wgrany domyślnie.

\subsubsection{Instalacja Django}
W projekcie zostało użyte Django w wersji 1.7.5, jednak od 01.04.2015 dostępna jest już wersja 1.8 LTS.

\begin{center}
\textbf{pip install Django==1.7.5}
\end{center}
\newpage
\subsection{Technologia MPI.}
MPI, czyli Message Passing Interface jest protokołem komunikacyjnym będącym standardem przesyłania komunikatów pomiędzy procesami programów równoległych działających na jednym, lub wielu komputerach. Ze względu na dominacje wykorzystywania tego modelu w klastrach komputerowych oraz superkomputerach, posłużył on jako główny protokół komunikacyjny w projekcie.
MPI jest specyfikacją biblioteki funkcji opartych na modelu wymiany komunikatów dla potrzeb programowania równoległego. Transfer danych pomiędzy poszczególnymi procesami programu wykonywanymi na procesorach maszyn będących węzłami klastra odbywa się za pośrednictwem sieci. Opisany schemat działania ilustruje Rysunek \ref{fig:mpi.jpg}.

\begin{figure}[h]
\centering
\includegraphics[scale=0.6]{Resources/mpi.jpg}
\caption{Ogólny schemat działania protokołu MPI.} 
\label{fig:mpi.jpg}
\end{figure}

Do poprawnego uruchomienia klastra obliczeniowego w technologii \textbf{MPI} potrzebne jest klika rzeczy:
\begin{itemize}
\item co najmniej dwie maszyny podłączone przez TCP/IP
\item wszystkie maszyny mają zainstalowaną system oparty na Linuxie
\item każda maszyna używa takiej samej architektury procesora, aby zapewnić poprawne działanie programu wykonywalnego
\end{itemize}

Aby móc uruchomić projekt, należy na wszystkich komputerach zainstalować biblioteki \textbf{OpenMPI}:

\begin{lstlisting}
sudo apt-get install openmpi-bin openmpi-common libopenmpi1 libopenmpi-dev
\end{lstlisting}

Dodatkowo, aby zapewnić sprawną komunikacją pomiędzy komputerem Master, a Slave’ami należy skonfigurować bez kluczowe połączenia. Do tego celu najlepszym rozwiązaniem jest użycie SSH, które umożliwi logowanie bez użycia hasła, wykorzystując do tego klucz publiczny, na przykład RSA. Podstawowy schemat działania wygląda następująco:

\begin{itemize}
\item \textbf{sudo apt-get install openssh-server} - instalacja serwera ssh na komputerze Master
\item \textbf{sudo apt-get install openssh-client} - instalacja klientów na komputerach Slave
\item \textbf{ssh-keygen –t rsa} - wygenerowanie kluczy na komputerze Master
\item \textbf{scp .ssh/id\_rsa.pub username@IP\_addr:~} - skopiowanie klucza publicznego do Slave'a
\item \textbf{cat id\_rsa.pub >> .ssh/authorized\_keys} - z poziomu Slave'a dodanie otrzymanych kluczy publicznych do  kluczy autoryzowanych
\end{itemize}

Mając skonfigurowane połączenie pomiędzy hostami, można skompilować program wydając polecenie: 

\begin{itemize}
\item \textbf{mpicc pi.c –o pi.out}
\end{itemize}

Po poprawnym i bezbłędnym skompilowaniu kodu, powinien pojawić się plik wynikowy z rozszerzeniem \textbf{.out}, który należy skopiować do wszystkich hostów:

\begin{itemize}
\item \textbf{scp pi.out username@ip\_addr:~}
\end{itemize}

Jeśli wszystkie hosty posiadają ten sam program, przed uruchomieniem klastra należy stworzyć i skonfigurować plik, zawierający konfigurację maszyn w klastrze. Tworzymy plik, np. .mpi\_hostfile, który należy uzupełnić według podanego schematu:

\begin{lstlisting}
#Hostfile dla Open MPI

#Master node, parametr 'slots=2' ustawiony, dlatego, ze node jest dwu-procesorowy
localhost slots=2

#Definiujemy slave'y, oraz dozwolona liczbe procesorow do uzycia
cichy@192.168.1.118 slots=2 max_slots=4
slawek@192.168.1.119 slots=8 max_slots=8
\end{lstlisting}

Jeśli wszystko przebiegło pomyślnie możliwe jest uruchomienie programu rozproszonego:
\begin{itemize}
\item \textbf{mpirun –np liczbaProcesorow –hostfile nazwaHostfile /sciezka/pi.out}
\end{itemize}

\subsection{CSS i HTML.}

\begin{figure}[h!]
\centering
\includegraphics[scale=0.3]{Resources/html5css3.png}
\caption{Logo HTML5 i CSS3.} 
\end{figure}

HTML5 – to język wykorzystywany do tworzenia i prezentowania stron internetowych. Jest rozwinięciem języka HTML 4 i jego XML-owej odmiany (XHTML 1). Został opracowany w ramach prac grupy roboczej WHATWG (Web Hypertext Application Technology Working Group) i W3C. Główną różnicą pomiędzy HTML 5 a XHTML 2 polega na tym, że HTML 5 jest pisane z myślą o kompatybilności wstecznej, podczas gdy XHTML 2 porzuciło kompatybilność. HTML5 poza dodaniem nowych elementów, usprawniających tworzenie serwisów i aplikacji internetowych, doprecyzowuje wiele niejasności w specyfikacji HTML 4, dotyczących przede wszystkim sposobu obsługi błędów. Niejasności co do sposobu, w jaki przeglądarki powinny obsługiwać błędy w kodzie HTML są jedną z podstawowych przyczyn, dla której wiele serwisów internetowych, napisanych z naruszeniem specyfikacji, w różnych przeglądarkach działa w inny sposób – w niektórych działając, w innych nie. Dzięki HTML-owi 5 obsługa błędów ma być ta sama we wszystkich przeglądarkach, czyli zły element będzie działać w każdej przeglądarce albo żadnej.
Bootstrap to framework CSS i Javascript, mający za zadanie ułatwić zaprojektowanie kompletnej strony internetowej w prosty sposób. Założeniem frameworka Twitter Bootstrap jest udostępnienie gotowych elementów, takich jak listy, slidery czy menu, które umożliwiają zaprojektowanie strony w spójny sposób w możliwie krótkim czasie. Sam framework oprócz bogatej biblioteki stylów CSS zawiera także elementy takie jak „karuzele” które obsługiwane są z użyciem języka JavaScript. Oprócz podstawowych elementów, takich jak przygotowane wstępnie górne menu nawigacyjne strony, czy też elementy list i formularzy dostępne są także różnego rodzaju slidery, boxy na treść wyróżnioną, boxy powiadomień czy też gotowe, uproszczone nagłówki

\subsection{Komunikacja grupy.}
Podczas realizacji projektu wykorzystane zostały narzędzia ułatwiające pracę w grupie. Stworzone zostało repozytorium, w którym umieszczane były kody źródłowe programu obliczającego liczbę Pi czy też apliakcji internetowej napisanej w frameworku Django. Ponadto na portalu Trello istniała tablica na której zamieszczane były zadania do zrealizowania. Umożliwiło to jasny podział zadań oraz bez problemową serializację efektów pracy niwelując liczbę konfliktów podczas równoległego tworzenia kodu przez różne osoby. Podczas realizacji projektu każdy członków z grupy dzięki dobrze zoorganizowanej tablicy Trello znał swoje zadania oraz terminy realizacji, które stanowiły motywację do pracy.

\subsubsection{GitHub.}

\begin{figure}[h!]
\centering
\includegraphics[scale=0.6]{Resources/GitHub_Logo.jpg}
\caption{Logo portalu GitHub.} 
\end{figure}

GitHub to serwis internetowy wykorzystywany masowo przy projektach programistycznych, korzystający z systemu kontroli wersji Git. Stworzony został już w 2008 roku przy wykorzystaniu języków Ruby on Rails oraz Erlang. Dzisiaj serwis obsługuje już kilka milionów repozytoriów, a wśród nich darmowe publiczne repozytoria oraz płatne prywatne. Darmowe repozytoria utrzymywane są na licencji open source. Podczas realizacji projektu został stworzony bardzo prosty tutorial korzystania z Git'a z poziomu konsoli poleceń. Tutorial opisuje korzystanie z podstawowych funkcjonalności systemu kontroli wersji , wyłączając bardziej zaawansowane techniki branch'owania i merge'owania projektów.

\begin{figure}[h!]
\centering
\includegraphics[scale=0.4]{Resources/Git_Logo.jpg}
\caption{Logo systemu kontroli wersji- Git.} 
\end{figure} 

Instrukcja do skonfigurowania i uruchomienia repozytorium opartego o system kontroli wersji Git na systemie Linux:

\begin{enumerate}
\item Instalacja najnowszej wersji gita za pomocą komendy:
\begin{center}
\textbf{sudo apt-get install git}
\end{center}

\item Drugi krok to ustawienie globalnych danych konta założonego na stronie GitHub:
\begin{center}
\textbf{sudo git config --global user.name "nazwaKonta" }
\end{center}

\item Następnie podajemy analogicznie maila na którego mamy zalożone konto na GitHub:
\begin{center}
\textbf{sudo git config --global user.email "email"}
\end{center}

\item Przed tym krokiem zostały już skonfigurowane dane konta.  Teraz należy je potwierdzić i polączyć się z interesującym nas repozytorium. Do nawiązania połączenia wykorzystywane są dwa sposoby : protokół HTTPS lub SSH. Tym razem wykorzystany będzie protokół HTTPS.

\item Prawdopodobnie po skonfigurowaniu danych konta Git pobierze automatycznie repozytoria, do których dane konto miało dostęp i umieści je w folderze Dokumenty.

\item W innym przypadku należy skopiować adres repozytorium ze strony portalu. Przykładowo:
\begin{center}
\textbf{https://github.com/syguts/AIiR0730obliczanieliczbyPI.git}
\end{center}

\item Nawiązanie połączenia poprzez komendę:
\begin{center}
\textbf{git clone https://github.com/syguts/AIiR0730obliczanieliczbyPI.git AIIR}
\end{center}
Gdzie "AIIR" to nazwa folderu, do którego zostanie pobrane repozytorium (wersja lokalna).
\end{enumerate}

Natomiast same edytowanie repozytorium sprowadza się do sekwencji:
\begin{enumerate}
\item \textbf{git fetch} - funkcja sprawdza wersje zdalnego repozytorium (jeżeli zdalna wersja jest identyczna co do lokalnej wówczas wyświetlany jest stosowny komunikat)
\item \textbf{git log} - metoda wyświetla listę wprowadzonych zmian
\item \textbf{git pull} - pobranie wprowadzonych zmian - zaaktualizowanie lokalnego repozytorium.
\item \textbf{git add} -  w przypadku dodania plików należy podłączyć je pod system kontroli wersji metodą git add nazwaPlikuFolderu
\item \textbf{git status} - metoda listująca zmiany wprowadzone przez użytkownika w wersji lokalnej od ostatniej aktualizacji
\item \textbf{git commit} - stworzenie commita zbierającego wprowadzone zmiany
\item \textbf{git push} - wypchanie wprowadzonych zmian na zewnątrz lokalnego repo, czyli dodanie do zdalnego repozytorium.
\end{enumerate}

\subsubsection{Trello.}

\begin{figure}[h!]
\centering
\includegraphics[scale=0.6]{Resources/Trello_Logo.png}
\caption{Logo portalu Trello.} 
\end{figure} 

Trello to aplikacja do zarządzania zadaniami. Opiera się ona na paradygmacie Kanban stworzonym w Japoni, w firmie Toyota, należącym do technik Agile. Trello to także produkt horyzontalny. Co to znaczy?. Oznacza to, że nie jest to produkt skierowany konkretnie do jednej grupy odbiorców, a wszystkich, którzy potrzebują danego narzędzia oraz chcących skorzystać z jego funkcji. Projekty w Trello zbudowane są w oparciu o tablice, na których to znajdują się listy zadań. Listy te pozwalają tworzyć karty (zadania), które w bardzo prosty sposób możemy sortować, czy przenosić pomiędzy różnymi listami za pomocą techniki drag-and-drop. Poszczególni użytkownicy mogą być przypisywani do kart oraz budować grupy, a grupy i konkretne projekty tworzą organizacje.Bardzo przydatną funkcjonalnością Trello są także powiadomienia, które spływają w formie wiadomości na adres email lub powiadomień (push) na urządzenia mobilne. Każdy użytkownik ma oczywiście kontrolę nad nimi oraz ich częstotliwością. Dzięki powiadomieniom można być informowanym o dyskusjach, w których pojawia się nazwa użytkownika, obserwowanych projektach, lub aktywnościach innych użytkowników. W ten sposób komunikacja następuje w błyskawicznym tempie.

\begin{figure}[h!]
\centering
\includegraphics[scale=0.2]{Resources/Trello_Zrzut.png}
\caption{Przykłądowy widok tablicy na portalu Trello.} 
\end{figure} 
\newpage

\section{Plan realizacji.}

\begin{enumerate}
\item Dokładna analiza algorytmu matematycznego na obliczanie liczby pi - Monte Carlo.
\item Zebranie informacji o metodach współbieżnych tego algorytmu lub też próba własnej implementacji.
\item Podział pracy na dwa zespoły:\\
Drużyna A - Cebula i Sternik:

\begin{itemize}
\item Implementacja algorytmu współbieżnego w języku C++
\item Nauka i zapoznanie z technologią MPI
\item Implementacj technologi MPI
\item Testy połączenia miedzy komputerami
\item Wykorzystanie w algorytmie współbieżnym
\item Testy silnika obliczeniowego - liczba pi
\end{itemize}

Drużyna B - Sygut i Cichuta:

\begin{itemize}
\item Nauka tworzenia stron internetowych z wykorzystaniem HTML5 i CSS
\item Zapoznanie z technologią frameworku Django
\item Utworzenie konta administracyjnego i użytkowników
\item Stworzenie strony internetowej
\item Testy- dodawanie użytkowników, administracja, wygląd na różnych przeglądarkach, pobieranie danych z formularzy
\item Postawienie aplikacji na serwerze i próba uruchomienia zdalnego
\end{itemize}

\item Połączenie silnika obliczeniowego z frontenedem i serwerem aplikacji.
\item Testowanie całego systemu pod różnymi aspektami- zmienność parametrów wejściowych dla algorytmu , liczba użytkowników chcących skorzystać z silnika ( kolejkowanie dostępności), płynność działania, górne ograniczenie czasowe, wygląd i interfejs aplikacji.
\item Tworzenie dokumentacji z podziałem prac według opracowanych obszarów systemu.
\item Przygotowanie prezentacji końcowej przedstawiającej efekty pracy oraz wnioski. Podsumowanie całego semestru prac - trudności i problemy oraz ich rozwiązania.
\end{enumerate}

\section{Implementacja silnika obliczeniowego.}

\subsection{Zrównoleglenie algorytmu}
Pierwszym krokiem podczas implementacji silnika obliczeniowego - programu obliczającego przybliżenie liczby Pi w oparciu o algorytm Monte Carlo była implementacja programu współbieżnego. Wykorzystana została w tym celu biblioteka systemu Linux - POSIX Threads.Biblioteka udostępnia jednolite API dla język C do tworzenia i zarządzania wątkami, jest dostępna w systemach m.in. Linux, FreeBSD, Windows. Implementacja miała na celu przygotowanie merytoryczne do implementacji algorytmu z wykorzystaniem technologii MPI. W tym celu zalecane było zapoznanie się z metodą zrównoleglenia tego algorytmu. Poniżej Listing programu z wykorzystaniem POSIX Threads.

\lstinputlisting[language=C++]{Code/wspol_pi.cpp}

Program pobiera od użytkownika ilośc punktów biorących udział w losowaniu oraz ilość wątków. Następnie w zalężności od liczby wątków obliczany jest zakres - przydział punktów dla każdego z wątków. Każdy z wątków zlicza ilość punktów znajdujących się w~kole. Po wykonaniu się funkcji wszystkich wątków, wątek macierzysty sumuje poszczególne liczniki i oblicza wartość przybliżenia Pi. 

\subsection{Zastosowanie technologii MPI}
Następnym krokiem podczas ralizacji silnika obliczeniowego było zaimplementowanie protokołu komunikacyjnego, czyli technologi MPI. Działa ona analogicznie co do metody współbieżnej przy czym wątki POSIX zostały zastąpione slave'mi MPI. Program posiada zdefiniowane na sztywno zmienne z biblioteki MPI określające liczbę stacji roboczych, które wczytuje z pliku konfiguracyjnego. Następnie identycznie jak w poprzednim programie w zależności od ilości punktów i maszyn wylicza zakresy dla każdego z wątków. Program przyjmuje w parametrach wywołania ilość punktów. Master po zakończeniu pracy slave'ów zlicza wynik cząstkowe i podstawia do wzoru obliczającego przybliżenie liczby Pi. Poniżej Listing programu: 

\lstinputlisting[language=C++]{Code/second_Program.c}

\subsection{Komunikacja z bazą danych aplikacji.}
Komunikacja z bazą danych aplikacji została zrealizowana poprzez wykorzystanie biblioteki języka C: mysql.h. Biblioteka zawiera metody umożliwiające zdalne połączenie z bazą danych MySQL poprzez adres IP i numer portu. Wymaga to również podanie danych użytkownika danej bazy. Zatem został zaimplementowany osobny program obsługujący akcje dokonywane na bazie danych i nasłuchujący pojawienie się zadań zleconych przez użytkowników aplikacji internetowej. Takie rozwiązanie umożliwiło stworzenie dodatkowej cechy systemu. Użytkownik jest w stanie określić ilość wątków na których ma zostać uruchomiony algorytm .Realizowane jest to poprzez wywołanie programu liczącego z poziomu komendy system z odpowienim parametrem kompilatora mpi open . Przykładowo :

\begin{center}
\textbf{system("mpirun -np lilosProcesow second iloscPunktow")}
\end{center}

Podsumowując program działa w procedurze:

\begin{enumerate}
\item Nawiązanie połączenia z bazą danych poprzez metodę \textbf{mysqlRealConnect}, która wymaga podania adresu bazy danych , nazwy użytkownika i jego hasła.

\item Program oczekuje czy zadanie pojawi się w bazie danych sprawdzając metodą biblioteki mysql pierwszy wiersz tabeli :

\begin{center}
\textbf{mysqlQuery(conn, "SELECT * FROM PI2Task limit 1")}
\end{center}

\item Po pojawieniu się zadania program pobiera dane zleconego zadania metodami : \textbf{mysqlUseResult(conn)}, gdzie conn jest uchwytem połączenia z bazą, \textbf{mysqlFetchRow(res)}, gdzie res to wszystkie dane.

\item Do tabeli wyników zostaje dodany wiersz sygnalizujący aplikację internetową o podjęciu zlecenia - włączenie dynamicznego progresu. Wiersz nie posiada wówczas uzupełnionej kolumny wyniku.

\item Natomiast  w tabeli zleconych zadań zostaje usunięty wiersz pobranego zadania. Dzięki temu tabela działa jak kolejka FIFO - First Input First Output. 

\item Nastepuje wywołanie programu liczącego z pobranymi wcześniej parametrami zadanymi przez użytkownika aplikacji internetowej. 

\item Program po zwróceniu wyniku przez program liczący aktualizuje wiersz w tabeli wyników. Następnie jeżeli coś znajduje się w kolejce od razu przystępuje do realizacji zadania . W innym przypadku oczekuje na zlecenie zadania. 
\end{enumerate}

Listing programu łączącego aplikację internetową z programem liczącym:

\lstinputlisting[language=C++]{Code/firstProgram.c}

\section{Aplikacja internetowa.}

\subsection{Tworzenie nowego projektu.}
W linii poleceń należy przejść do katalogu, w którym będzie znajdował się kod, następnie wpisując:\\
\textbf{django-admin.py startproject PI}\\
Utworzony zostanie katalog z projektem:
PI/\\
init.py\\
manage.py\\
settings.py\\
urls.py

\begin{itemize}
\item \textbf{init.py: }Pusty plik informujący Pythona o tym, że katalog nadrzędny powinien być traktowany jako pakiet Pythona 
\item \textbf{manage.py: }Działające z linii poleceń narzędzie, które pozwala na interakcję z~projektem Django na różne sposoby.
\item \textbf{settings.py}Ustawienia/konfiguracja dla tego projektu Django.
\item \textbf{urls.py}Deklaracja adresów URL dla tego projektu Django; “mapa serwisu” Twojej strony zbudowanej w oparciu o Django.
\end{itemize}

Przechodząc do katalogu PI, można sprawdzić czy server deweloperski działa, w tym celu należy wpisać:

\begin{center}
\textbf{python manage.py runserver}
\end{center}

\subsection{Ustawienia bazy danych.}
Django można używać bez bazy danych, jednak dostarczony jest wraz z systemem ORM (odwzorowanie relacyjno-obiektowe), za pomocą, którego można definiować tabele w bazie danych oraz pracować na nich pisząc i operując na klasach w Pythonie. Django współpracuje z następującymi serwerami bazodanowymi: PostgreSQL, MySQL, Oracle i~SQLite. Ze względu na łatwość obsługi wybrany został MySQL, przez co należało doinstalować MySQLdb:

\begin{center}
\textbf{sudo apt-get install python2.7-mysqldb}
\end{center}

Co wymagało również zmiany silnika bazy danych w ustawieniach aplikacji na „django.db.backends.mysql” oraz z wnętrza powłoki bazy danych utworzenia 
potrzebnej bazy danych. Ze względu, że w pliku settings.py INSTALLEDAPPS zawarte są domyślne aplikacje, należy użyć komendy:

\begin{center}
\textbf{python manage.py syncdb, która dla w/w aplikacji tworzy tabele w bazie.}
\end{center}

\subsection{Tworzenie modeli.}
W katalogu PI, aby zacząć pracę nad projektem należy wprowadzić następujące polecenie:

\begin{center}
\textbf{python manage.py startapp PI2}
\end{center}

Co utworzy podkatalog PI2, w którym będzie się znajdować struktura docelowej aplikacji. W pliku models.py zostały stworzone następujące klasy:

\lstinputlisting[language=Python]{Code/models.py}

W tabeli Task znajdują się zadania, zlecone przez użytkowników, które oczekują w kolejce do pobrania i obliczenia przez silnik. 

\begin{itemize}
\item userID – ID usera zgłaszającego zadanie
\item numberOfPoints – liczba punktów
\item numberOfThreads – liczba wątków, na których zostaną wykonane obliczenia
\end{itemize}

W pliku forms.py zostały dodane klasy obsługi rejestracji oraz zlecenia zadania:

\lstinputlisting[language=Python]{Code/forms.py}

Formy pobierające dane zawierają następujące ograniczenia:
\begin{itemize}
\item Liczba punktów nie może być mniejsza od 1
\item Liczba wątków musi się znajdować w przedziale [1-12]
\end{itemize}

W pliku admin.py została dodana możliwość przeglądania przez administratora profilów użytkowników, zleconych zadań oraz skończonych zadań.

\begin{figure}[h!]
\centering
\includegraphics[scale=0.6]{Resources/screenAdmin.png}
\caption{Panel administratora.} 
\end{figure} 

\subsection{Tworzenie widoków wraz z obsługą żądań}
W django do generowania widoków wykorzystywane są pliki z rozszerzeniem ".html". W aplikacji wykorzystano strony napisane w jezyku html w wersji 5 oparte o technologie twitter bootstrap.
Aby w django utworzyć strone internetową z utworzonymi plikami "html", należy utworzyć folder w katalogu projektu o nazwie "templetes". W folderze tym umieszcza się wszystkie pliki statyczne tzn. dodatkowe biblioteki javascript (z rozszerzeniem: jar, js) pliki styli "css" oraz wszystkie pliki "html" tworzonego portalu.
Nastepnie tworzy się plik urls.py w którym umieszcza się scieżki do plików html oraz podaje obsługujące je metody jezyka python zdefiniowane w pliku views.py:
Listening pliku urls.py który wykorzystano w projekcie:

\lstinputlisting[language=Python]{Code/urls.py}

W pliku views.py zdefiniowane są wszystkie kontrolery obsługujące scieżki url oraz request'y, gdyż korzystając z technologi ajax można wywołać dany kontroler requestem POST lub GET.
Listening prezentujący fragment pliku views.py który wykorzystano w~projekcie:

\lstinputlisting[language=Python]{Code/views.py}

\subsection{Zrzuty ekranów aplikacji internetowej}

\begin{center}
\begin{figure}[h!]
\centering
\includegraphics[scale=0.2]{Resources/mainSite.jpg}
\caption{Główna strona aplikacji internetowej.} 
\end{figure} 
\end{center}

\begin{center}
\begin{figure}[h!]
\centering
\includegraphics[scale=0.7]{Resources/registerSite.png}
\caption{Widok rejestracji.} 
\end{figure} 
\end{center}

\begin{center}
\begin{figure}[h!]
\centering
\includegraphics[scale=0.4]{Resources/logInSite.png}
\caption{Widok zaloowanego użytkownika.} 
\end{figure} 
\end{center}

\begin{center}
\begin{figure}[h!]
\centering
\includegraphics[scale=0.4]{Resources/adminLoginSite.png}
\caption{Widok zaloowanego użytkownika.} 
\end{figure} 
\end{center}
\newpage
\section{Testy.}

Aplikację testowaliśmy pod kątem złożoności czasowej w stusnku do liczby punktów. Testy zostały przeprowadzone na 3 stacja roboczych o podanej konfiguracji:

\begin{enumerate}
\item \textbf{Stacja robocza 1}
\begin{itemize}
\item Procesor: Core2Quad Q9400 (6M Cache, 2.66 GHz, 1333 MHz FSB)
\item Pamięć: DDR3 4GB 1333MHz
\item System: Linux Ubuntu 14.04
\end{itemize}
\item \textbf{Stacja robocza 2}
\begin{itemize}
\item Procesor: Core2Duo T6500 (2M Cache, 2.10 GHz, 800 MHz FSB ) 
\item Pamięć: DDR3 4GB 1333MHz
\item System: Linux Ubuntu 14.04
\end{itemize}
\item \textbf{Stacja robocza 3}
\begin{itemize}
\item Procesor: AMD E-450 (1M Cache, 1.65 GHz)
\item Pamięć: DDR3 4GB 1333MHz
\item System: Linux Ubuntu 14.04
\end{itemize}
\end{enumerate}

\begin{center}
\begin{figure}[h!]
\centering
\includegraphics[scale=0.85]{Resources/tabela.jpg}
\caption{Tablerayczne zestawienie wyników.} 
\end{figure} 
\end{center}

\begin{center}
\begin{figure}[h!]
\centering
\includegraphics[scale=0.77]{Resources/wykres.jpg}
\caption{Graficzne przedstawienie wyników.} 
\end{figure} 
\end{center}

Jak widać z przedstawionych wyników, złożoność obliczeniowa zaimplementowanego algorytmu jest liniowa w stosunku do liczby punktów oraz liczby procesorów. Ze wzlędu na ograniczenia typu \textbf{unsigned int} zakres wpisywanych punktów mieści się w przedziale od 0 do około 2 100 000 000 punktów. 

\newpage
\section{Podsumowanie i wnioski.}

W ramach kursu Aplikacje Internetowe i Rozproszone został zrelizowany projekt systemu rozproszonego odpowiedzialnego za obliczanie przybliżenia dziesiętnego liczby Pi algorytmem Monte Carlo. System ten składa się z kilku modułów takich jak: silinik obliczeniowy napisany w języku C, aplikacja internetowa z użyciem frameworku języka Python - Django, baza danych MySQL, komunikacja między stacjami roboczymi zrealizowana przy pomocy technologii MPI. Ostatecznie zbudowany system spełnia założenia projektowe i realizuje cele całego kursu.

 Podczas rozwoju projektu grupa zdobyła wiedzę z zakresu pracy nad złożonym problemem inżynieryjskim w środowisku zespołu programistycznego. Do ułatwienia tego zadania wykorzystane zostały narzędzia do komunikacji i współdzielenia zasobów projektu. Mowa tutaj o zdalnym repozytorium opartym o system kontroli wersji Git na platformie GitHub. Planowanie pracy jak i przydział zadań odbywał się natomiast przy wykorzystaniu multimedialnej tablicy na serwisie Trello. 
 
System działa poprawnie i spełnia wszystkie założone funkcjonalności przy respektowaniu pewnych ograniczeń, które zostały zaimplementowane. Użytkownik może maksymalnie wykorzystać 12 wątków w obecnej konfiguracji maszyn roboczych. Ograniczenie uległoby zmianie przy jakimkolwiek zmianie konfigarcji. Innym ograniczeniem jest maksymalna liczna punktów rozpatrywanych podczas pracy algorytmu. Wynika to z architektury programu - zakresu zmiennej przechowującej dane. 

W ramach możliwego rozwoju aplikacji należałoby przystosować system, a w szczególności silnik obliczeniowy do przyjmowania większego zakresu punktów oraz optymalizacji wymiany informacji między stacjami roboczymi. Inaczej mówiąc zwiększenie automatyzacji wgrywania oprogramowania silnika na każdą z maszyn.

Komunikacja w zespole odbywała się bez zarzutów o czym może świadczyć uporządkowana tablica na portalu Trello oraz systematyczna praca, którą można odtworzyć na zdalnym repozytorium.

\section{Bibliografia }

\begin{enumerate}
\item Bootstrap Tworzenie Interfejsów Stron WWW - Syed Fazle Rahman
\item Parallel Programming with MPI - Peter Pacheco
\item Documentation of Django - https://docs.djangoproject.com/en/1.8/
\item http://www.zsk.ict.pwr.wroc.pl/zsk/pracownicy/hmaciejewski
\item Tango with Django Book - https://github.com/leifos/tango\_with\_django\_book
\end{enumerate}

\end{document}
